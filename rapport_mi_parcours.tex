\documentclass[a4paper, 12pt,twoside]{article}
\usepackage[utf8]{inputenc}
%\usepackage[T1]{fontenc}
\usepackage[frenchb]{babel}

\usepackage{lmodern}
\usepackage{textcomp}
\usepackage{ifthen,amsmath,amsfonts,amssymb,graphicx}
\usepackage{enumitem}
\usepackage{multicol}
\usepackage[notes,
            titlepage,
            a4paper,
            pagenumber,
            sectionmark,
            twoside,
            fancysections]{polytechnique}
\usepackage[colorlinks=true,
            linkcolor=black,%bleu303,
            filecolor=red,
            urlcolor=bleu303,
            bookmarks=true,
            bookmarksopen=true]{hyperref}

\title{Projet Scientifique Collectif}
\author{Pierrick \textsc{Allègre} \\
        Gustavo \textsc{Castro} \\
        Clément \textsc{Durand} \\
        Felipe \textsc{Garcia} \\
        Alexandre \textsc{Harry} \\
        Francisco \textsc{Ribeiro Eckhardt Serpa} \\
        Pierre-Alexandre \textsc{Thomas} \\
        Guillaume \textsc{Vizier} \\
        Tuteur~: Thomas \textsc{Clausen} -- Coordinateur~: Franck \textsc{Nielsen} \\
        Cadre~: Capitaine Angélique \textsc{Aubois} \\}
\subtitle{Rapport de mi-parcours \\ INF07}
\date{\today}


\begin{document}
\maketitle
\renewcommand{\baselinestretch}{1.1}
\setlength{\parskip}{0.5em}
\tableofcontents
\clearpage

\addcontentsline{toc}{section}{Présentation succincte du sujet}
\section*{Présentation succincte du sujet}
    Objectif final
    
    Objectif de la première partie d'année

\section{Le point sur nos résultats}

    \subsection{Mise en place des moyens de communication}

    Nous avions décidé d'utiliser les moyens de communication suivants~:
    \begin{description}
        \item[Git~:] Pour le développement collaboratif de nos programmes
        \item[Slack~:] Pour utiliser un moyen de communication instantanée, plus polyvalent qu'IRC. En effet, nous l'utilisons aussi pour surveiller l'état des dépôts \verb!git!. De plus, cela nous permet d'éviter Facebook, qui en plus d'être peu pratique, est totalement imperméable au concept même de sécurité.
        \item[Trello~:] Qui permet de clarifier l'état actuel du projet en termes de tâches à accomplir.
        \item[FramaDate~:] Pour permettre une meilleure organisation, notamment trouver une plage horaire compatible avec les disponibilités de tous, incluant le tuteur et notre coordinatrice DFHM.
    \end{description}

    Nous avons constaté que des logiciels adaptés permettent une organisation claire, et que la difficulté venait plutôt d'un tout autre côté.

    \subsection{Compréhension du réseau}

    Nous avons commencé notre projet par une phase de recherche,  notamment au sujet des différents protocoles existants. Nous avons décidé alors de nous concentrer sur les protocoles suivants~:
    \begin{description}
        \item[DNS~:] Ce protocole donnant aux utilisateurs les adresses des sites qu'ils veulent consulter, il semble propice à une éventuelle exploitation par un tiers pour réaliser une attaque, et il était donc pertinent pour nous de l'étudier.
        \item[TCP~:] Ce protocole permettant un échange suivi entre deux entités du réseau, il est utilisé notamment comme support du protocole HTTP, servant à la transmission de pages web.
    \end{description}

    Notre  utilisation de \verb!dig! nous a permis de mieux comprendre l'architecture du réseau de l'École polytechnique, ainsi que le fonctionnement du protocole DNS.

    \subsection{DNS poisoning}

    Après cette phase de recherche sur les différents protocoles, et les réseaux en général, nous avons commencé à nous intéresser plus spécifiquement au DNS poisoning. Nous avons découpé le travail en deux parties~: une partie observation, et une partie action. Nous avons donc écrit avec \verb!scapy! (\verb!python!) deux programmes~:
    \begin{itemize}[label=\color{bleu303}\textbullet{}]
        \item L'un qui analyse le trafic, détecte les requêtes DNS, et affiche les résultats, avec des informations sur le paquet (donc la requête).
        \item L'autre répond à chaque requête DNS par un paquet DNS fabriqué à la main.
    \end{itemize}

    Ces deux programmes sont à l'heure actuelle opérationnels, au sens où lorsqu'ils tournent et qu'un utilisateur du réseau sur lequel nous sommes envoie une requête pour consulter un site internet, notre programme simule une réponse du serveur DNS destinée à envoyer l'utilisateur sur une page web choisie par le programme. Cela permettrait par exemple à l'attaquant de rediriger l'utilisateur sur une fausse page d'identification destinée à récupérer ses identifiants.

    \noindent Nous avons cependant rencontré un certain nombre de difficultés, notamment~:
    \begin{itemize}[label=\color{bleu303}\textbullet{}]
        \item Détecter un paquet DNS ne nous a pas posé de réel problème, mais il fallait détecter plus précisément les requêtes DNS, ainsi que récupérer les adresses IP, ports, etc. pour pouvoir créer une réponse adaptée. Les recherches que nous avions faites sur les différents protocoles nous ont permis d'identifier assez facilement les parties d'un paquet IP qui identifiaient les requêtes et réponses DNS, ainsi que celles qui contenaient leurs données.
        \item Nous réussissions à envoyer un paquet DNS, mais celui-ci semblait n'apparaître nulle part comme s'il n'était pas reçu par la cible. Une recherche plus poussée nous a permis de constater qu'il apparaissait en fait comme non-valide (\emph{broken packet}) pour \verb!wireshark! (qui était le seul de nos logiciels à le voir)~: la checksum était fausse, et comme c'est l'élément qui assure l'ordinateur de l'intégrité du paquet, il était naturellement refusé par l'utilisateur. Nous avons donc modifié la façon dont nous créons le paquet DNS réponse~: \verb!scapy! calcule automatiquement les checksums, mais ne les recalcule pas en cas de modification du contenu du paquet, c'est pourquoi nous avons simplement créé nos paquets à partir de rien plutôt que de modifier des paquets copiés.
        \item La commande \verb!dig! ne reconnaissait pas notre paquet réponse. Une théorie quant à la cause de ce problème fut l'absence d'autorité dans les données de nos premiers paquets réponse. Nous avons donc ajouté les autorités, serveurs de l'École polytechnique, dans la section “autorités” du paquet réponse, section a priori nécessaire pour que l'utilisateur accepte une réponse. Nous avons vérifié par ailleurs que le paquet était accepté par \verb!wireshark! qui en effet ne signale plus de paquet brisé.
        \item Cela n'ayant pas résolu le problème, nous avons connecté nos ordinateurs sur un réseau Wi-Fi personnel (différent de celui de l'École polytechnique), et avons fait attention à interroger nos ordinateurs personnels, en passant par la bonne carte réseau. Nous avons eu alors des résultats positifs.

        Le fait d'utiliser des réseaux personnels permet tout d'abord d'éviter de cibler ou écouter des ordinateurs extérieurs à notre groupe, ce qui serait illégal et ne serait pas non plus cohérent avec nos objectifs.

        \item La rapidité des serveurs DNS est un réel problème, auquel nous ne nous sommes pas encore attelés. En effet, pour réussir l'attaque, il faut répondre avant les vrais serveurs DNS~: la première réponse est la réponse enregistrée. Il se trouve que nos premiers ordinateurs de test étaient extrêmement lents, ce qui nous a permis de constater ce problème. En utilisant des ordinateurs plus performants nous nous sommes rapprochés des résultats espérés, suffisamment pour avoir un programme fonctionnel. En revanche, dans nos phases de test, nous interrogeons un serveur DNS qui «~n'existe pas~» ce qui permet de ne pas avoir ce problème de temps de réponse et de se concentrer sur d'autres éléments.
    \end{itemize}

    \subsection{Interruption d'une communication TCP établie}


  Pour assurer l'efficacité de l'attaque précédente, il faut faire en sorte qu'il y ait des requêtes DNS. Pour cela, il faut interrompre les connexions TCP existantes. Nous avons écrit un programme, toujours avec \verb!Scapy!, qui envoie un paquet demandant la fin de la communication au demandeur (paquet FIN).


       Dans le cas général, une connexion TCP commence par un three-way handshake (il existe aussi l’ouverture simultanée, mais elle est beaucoup moins fréquente)~: le client envoie un paquet SYN (synchronize) au serveur~; le serveur répond avec un paquet SYN+ACK (acknowledgement), signalant qu'il a bien reçu la demande de connexion et qu'il est prêt à louvrir de son côté~; enfin, le client renvoie un paquet ACK et la connexion est finalement établie. Les paquets portent aussi des numéros de séquence, qui sont importants pour éviter toute confusion puisque la connexion d'un même client peut se fermer et rouvrir en succession rapide ou alors être interrompue avec perte de mémoire (il faut la rétablir à partir de zéro).
    
      Après l'établissement, client et serveur échangent des données jusqu'à ce que l'un des deux demande de mettre un terme à la connexion, lorsqu'ils n'ont plus de données à envoyer. D'abord, nous avions l'idée d'imposer cette opération prématurément en envoyant un paquet FIN. Notre programme analyse le trafic, filtre les paquets TCP et, pour les paquets SYN trouvés, renvoie un paquet FIN fabriqué à la main.
    
      De même que pour les paquets DNS, la détection des paquets TCP-SYN, avec les connaissances de sa structure, et la fabrication des paquets réponse, non pas par modification des paquets reçus, mais par création de nouveaux paquets avec \verb!Scapy! pour éviter des problèmes comme celui de la somme de contrôle (checksum), ne nous ont pas posé de grandes difficultés. Cependant, ce programme est trop lent pour réellement interrompre la communication~: la vraie réponse arrive toujours avant la nôtre, qui est invariablement ignorée. Nous envisageons donc deux possibilités de travail~:
  \begin{itemize}[label=\color{bleu303}\textbullet{}]
      \item Réécrire ce programme dans un langage plus bas-niveau, afin d'améliorer le temps de réaction de notre programme, voire le faire plus vite que les serveurs~: nous avons choisi le langage C.
      \item Anticiper les réponses suivantes~: s'il nous est impossible de prendre de vitesse la première réponse, nous allons donc essayer de prendre la place de la deuxième, troisième ou énième réponse. Cela nous force à analyser plusieurs échanges de paquets TCP, pour comprendre et être en mesure de prédire les incrémentations futures. En effet, à chaque échange de paquets, les numéros de séquence sont incrémentés dans les paquets retournés. Si nous sommes capables de prédire ses valeurs après n échanges, nous avons résolu notre problème~: même si notre réponse est plus lente, nous pourrons la minuter afin qu'elle s'ajuste à la bonne position.
  \end{itemize}


  C'est sur ces points que nous allons travailler dans les semaines à venir.
\newpage

\section{Actualisation des objectifs}

    \begin{itemize}[label=\color{bleu303}\textbullet{}]
        \item Accélérer le programme pour le DNS poisoning
        \item Accélérer le programme TCP et anticiper sur les réponses suivantes
        \item Comment bloquer les attaques ci-dessus~?
    \end{itemize}

    Nous avons manifestement pris du retard sur les objectifs initiaux~: nous souhaitions avoir fini la phase «~attaque~» du projet pour le rendu de ce rapport, or l'une des deux parties de cette phase n'est pas suffisamment efficace pour être utilisée dans une attaque réelle. Nous devrons donc revoir nos objectifs à la baisse.

    Notre objectif final est maintenant de concevoir une protection contre une attaque de type DNS poisoning, ou au moins d'empêcher que la connection TCP soit interrompue fallacieusement.

    Cependant, il faudra tout d'abord finir ce que nous avons commencé~: optimiser le code de l'interruption de connexion TCP en abaissant le niveau du langage, et en essayant d'anticiper sur les échanges à venir.

    Puis nous prendrons contact avec la Direction des Systèmes d'information pour avoir des informations sur le dispositif mis en place peu avant les vacances, et qui bloque la vue des paquets en transit qui ne nous sont pas destinés~: la manipulation rend le mode promiscuous inefficace. Par ailleurs, nous devrons engager une nouvelle phase de recherche ce mois-ci, afin d'avoir suffisamment de connaissances sur l'état de l'art pour développer un outil capable de contrer ceux que nous avons créés jusqu'à présent.

\end{document}
